\subsection*{Exercise 01 (Cecilia)}

Denote the space containing the sequence $ a_n $ as $ X $, that space must have the $ || $ operator defined. We further assume that $ || $ operator satisfies the triangle inequality, it will work, for example, with real or complex numbers.

Suppose the sequence $ a_n $ converges, for any $ \epsilon > 0 $, there exists $ N \in \mathbb{Z} $ and $ L \in X $ such that when $ n \ge N $, $ |a_n -  L| < \epsilon $. 

Now consider $ ||a_n| - |L|| $, by exercise 1.13, we have $ ||a_n| - |L|| \le |a_n - L| < \epsilon $, so the same $ \delta $ can be used to prove that $ |a_n| $ converges.

The converse is not true, consider the sequence $ a_n = (-1)^n $, the sequence does not converge, but the sequence $ |a_n| = 1 $ converges.