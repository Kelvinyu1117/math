\subsection*{Exercise 22 (Cecilia)}
Consider the set $ E \subset \mathbb{R}^n $ of points having rational coordinates.

$ E $ is countable because it is a finite Cartesian product of countable sets.

For any point $ p = (p_1, p_2, \cdots, p_n )$ is $ \mathbb{R}^n $, for any neighborhood $ N_p(r) $ of $ p $, we claim that there exists a point $ q \in E $ such that $ q \in N_p(r) $.

Indeed, by the Archimedean property, there exists a rational number $ q_i $ such that $ p_i - r < q_i < p_i $ all $ i = 1, 2, \cdots, n $, and so $ q = (q_1, q_2, \cdots, q_n) \in N_p(r) $.

Therefore $ p $ is a limit point of $ E $, and so $ E $ is dense in $ \mathbb{R}^n $.

$ E $ is both countable and dense, and so $ \mathbb{R}^n $ is separable.