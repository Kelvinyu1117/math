\subsection*{Exercise 25 (Cecilia)}
Since $ K $ is compact, consider these open covers of the space as follows:

For each $ n \in \mathbb{N} $, define the open cover of the space as $ \{ N_{\frac{1}{n}}(k) \mid k \in K \} $.

Each of these open cover has a finite subcover by compactness of $ K $, so the union of these finite subcovers is a countable set.

We claim that the centers of these finite subcovers $ C $ form a countable dense subset of $ K $.

To see this, consider any point $ k \in K $ and any neighborhood $ N_r(k) $ where $ r > 0 $, there exists a rational number $ 0 < \frac{p}{q} < r $ by the Archimedean property, and so $ 0 < \frac{1}{q} < r $. The point $ k $ must be covered by an open set in the finite subcover of radius $ \frac{1}{3q} $, and that must be a subset of $ N_r(k) $. In other words, there exists a point $ c \in C $ such that $ c \in N_r(k) $, meaning $ k $ is a limit point of $ C $, and so $ C $ is dense in $ K $.

Therefore $ K $ is separable.

We claim that these finite subcovers forms a base of $ K $.

To see this, consider any point $ k \in K $ and any neighborhood $ N_r(k) $ where $ r > 0 $, there exists a rational number $ 0 < \frac{p}{q} < r $ by the Archimedean property, and so $ 0 < \frac{1}{q} < r $. The point $ k $ must be covered by an open set in the finite subcover of radius $ \frac{1}{3q} $, and that must be a subset of $ N_r(k)$. Therefore the finite subcovers forms a base of $ K $.

Therefore $ K $ has a countable base.