\subsection*{Exercise 13 (Kelvin)}

\begin{equation}
    ||x| - |y|| \le |x - y| \implies
        \begin{cases}
            |x| - |y| \le |x - y| & \text{if $|x| - |y| \ge 0$}\\
            |y| - |x| \le |x - y| & \text{if $|x| - |y| < 0$}
        \end{cases}
\end{equation} \\
Consider if $|x| - |y| \ge 0$: \\
$|x| = |(x - y) + y| \le |x - y| + |y| \implies |x| - |y| \le |x - y|$ (By theorem 1.33 (e)) \\ \\
Consider if $|y| - |x| < 0$: \\
$|y| = |(y - x) + x| \le |y - x| + |x| \implies |y| - |x| \le |x - y|$ (By theorem 1.33 (e)) \\ \\
Therefore, $||x| - |y|| \le |x - y|$.
    
